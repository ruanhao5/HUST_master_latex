
%%% Local Variables:
%%% mode: Xelatex
%%% TeX-master: t
%%% End:

\ctitle{标题:宋体,英文 Times New Roman,一号,加粗,不超 30 字\\
\textcolor{red}{(中英文标题、学科专业、导师姓名正确、一致)}}

\xuehao{M2019xxxxx} \schoolcode{10487}
\csubjectname{XXXXX} \cauthorname{XXX}
\csupervisorname{XXX} \csupervisortitle{教授}
\defencedate{202X~年~X~月~X~日} \grantdate{}
\chair{}%
\firstreviewer{} \secondreviewer{} \thirdreviewer{}

\etitle{English Title,Times New Roman,小二号,实词的首字母大写\\
\textcolor{red}{中英文标题、学科专业、导师姓名正确、一致}}
\edegree{the Master Degree in Engineering} %(工学硕士)
%\edegree{the Master Degree in Science}     %(理学硕士)
%\edegree{the Professional Master Degree}   %(专业学位)
\esubject{Control Science and Engineering}
\eauthor{xxx(中文习惯,姓在前且姓全部大写)}
\esupervisor{Prof. Xxxx}

%定义中英文摘要和关键字
\cabstract{
摘要是学位论文极为重要、不可缺少的组成部分,它是论文的窗口,并频繁用于国内外资料交流、情报检索、二次文献编辑等。其性质和要求如下:
\begin{enumerate}
    \item 摘要即摘录论文要点,是论文要点不加注释和评论的一篇完整的陈述性短文,具有很强的自含性和独立性,能独立使用和被引用。
    \item 博士学位论文的摘要应包含全文的主要信息,并突出创造性成果。
    \item 内容范围应包含以下基本要素:
    \begin{enumerate}
        \item 目的:研究、研制、调查等的前提、目的和任务以及所涉及的主题范围。
        \item 方法:所用原理、理论、条件、对象、材料、工艺、手段、装备、程序等。
        \item 结果:实验的、研究的、调查的、观察的结果、数据,被确定的关系,得到的效果、性能等。
        \item 结论:结果的分析、研究、比较、评价、应用;提出的问题,今后的课题,建议,预测等。
        \item 其他:不属于研究、研制、调查的主要目的,但就其见识和情报价值而言也是重要的信息。
    \end{enumerate}
    \item 摘要的详简度视论文的内容、性质而定,\textcolor{red}{硕士学位论文摘要一般为500-600汉字。}
    \item 摘要及全文中均建议不出现“我们”等字样。摘要中主语(作用)常常省略,因而一般使用被动语态;应使用正确的时态,并要注意主、谓语的一致,必要的冠词不能省略。
    \item 一般不用图、表、化学结构式、计算机程序,不用非公知公用的符号、术语和非法定的计量单位。
    \item 摘要中一般不使用缩写词,若实在需要,在第一次使用前,需给出中文全称(缩写词);在使用英文缩写词之前,需给出英文全称(英文全称,缩写词),再次出现时可以采用中文或英文缩写词。
    \item \textcolor{red}{关键词应有3至8个,另起一行置于摘要下方,领域从大到小排列。关键词之间用分号隔开,最后一个关键词后面无标点。}
    \item 摘要、关键词采用中文宋体;英文Times New Roman;小四号。
    \item 应有与中文摘要和关键词相对应的英文摘要和关键词。英语摘要用词应准确,使用本学科通用的词汇。 
\end{enumerate}}

\ckeywords{关键词1;关键词2;关键词3}

\eabstract{This is abstract.

英文摘要字体为Times New Roman,小四,1.5倍行距。

英文摘要和关键词应与中文相对应。英语摘要用词应准确,使用本学科通用的词汇;摘要中主语(作用)常常省略,因而一般使用被动语态;应使用正确的时态,并要注意主、谓语的一致,必要的冠词不能省略。}

\ekeywords{Keyword1, Keyword2, Keyword3}
