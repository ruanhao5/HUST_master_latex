
%%% Local Variables:
%%% mode: latex
%%% TeX-master: t
%%% End:

\chapter{系统与控制理论类论文*}
\label{cha:command}


\section{引言(引言标题可选)}
\label{sec:cover}

由于绪论中已对全文相关的研究背景和进展做了综述,因此,每章的引言中,请用1页左右的版面写一个导引,简要说明本章研究的背景或动机,以起到承上启下的作用,不宜太长。

引言的最后一段,说明本章的主要内容,如拟基于什么理论或方法,针对什么问题开展研究。注意:这里不能给结论。

若少于1个页面时,建议省略标题“引言”,直接在章的标题下写上几段话即可。

本参考模板以系统与控制理论类论文为例,各院系可根据本学科特点另行约定。

若学位论文属系统与控制理论类论文,且论文所用的基本的理论与方法基本一样,但具体应用场景不同,此时可考虑在第二章用整章的篇幅来描述,这样后面的章节就无需描重复描述相同内容,以避免冗余。

若使用的研究理论与方法并非完全相同时,建议在各章中分别介绍相关理论与方法(预备知识)、问题描述、控制策略与闭环系统分析,数值例子与实验分析,最后是小结。


\section{预备知识(可选,标题可自选)}
\label{sec:font}

\subsection{预备知识1}
陈述后续所用基本理论与方法,一般应给出相关内容来源,可引用相关文献,本部分篇幅应严格控制。给出预备知识的目的是方便大同行专家阅读理解学位论文,但要注意避免常识性或教科书基本知识的陈述。譬如,学位论文中用到深度神经网络模型,只需要叙述清楚所用的深度神经网络模型,而不需要从生物神经元、感知器模型等基础知识开始陈述。

\subsection{预备知识2}
预备知识2的相关叙述……

\subsection{预备知识3}
预备知识2的相关叙述……


\section{问题的描述(请拟定具体的题目)}
清晰的叙述本章所要研究的内容,并给出相应的数学模型与其基本假设。例如,

这里我们考虑一个由N个跟随者和一个编号为0的领导者构成的多自主体网络。其中,跟随者的动力学由如下动力学描述
\begin{equation}
\sum_{i=1}^{\left[ \frac{n}{2}\right]} \binom{x_{i,i+1}^{i^2}}
{\left[\frac{i+3}{3} \right]} \frac{\sqrt{\mu(i)^{\frac{3}{2}}
(i^2-1)}} {\sqrt[3]{\rho(i)-2}+\sqrt[3]{\rho(i)-1}}
\end{equation}
其中第i个自主体的状态$x_i (t)\in R$,f$(x_i (t))$代表非线性连续可微函数,$U_i$表示节点i的控制输入。

\textcolor{red}{(论文所有公式的按章节顺序编号,并引用)}

……

作为章节的二级标题,若直接采用“问题的描述”,则在目录中看不到有效信息。为避免出现毫无辨识度的标题,建议将所得的结果或者研究的问题作为二级标题,但也不要一幅图一节,也可以考虑将问题描述与后续的控制器设计整合一起作为一节,各节篇幅长短不宜悬殊太大。

\section{控制器设计与闭环系统分析(请根据所设计的控制器特点自行拟定具体的题目)}

\subsection{控制器设计(标题可自选)}
控制器设计的详细叙述……

\subsection{闭环系统稳定性分析(标题可自选)}
稳定性分……

\section{数值仿真(请拟定具体的题目)}
本节中…

\section{本章小结}
\label{sec:theorem}

本章主要介绍系统与控制理论类论文正文章节的框架结构。在每章的最后,都需要对该章的内容进行小结,不宜太长,建议1/2-2/3页版面较好。主要小结一下本章用什么理论或方法、做了什么事、得到的重要结果或结论。
